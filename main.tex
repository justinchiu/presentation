\documentclass{beamer}
% \mode<presentation>
\setbeamertemplate{navigation symbols}{}
\let\tempone\itemize
\let\temptwo\enditemize
\renewenvironment{itemize}{\tempone\addtolength{\itemsep}{0.5\baselineskip}}{\temptwo}
\usepackage{beamerthemeshadow}
\usepackage[normalem]{ulem}
\usepackage{tikz}
\usepackage{tikz-dependency}
\usetikzlibrary{shapes.arrows, positioning, fit, bayesnet}

\tikzset{
    myarrow/.style={
        draw,
        fill=gray
        single arrow,
        minimum height=3.5ex,
        single arrow head extend=1ex
    }
}
\newcommand{\arrowup}{%
\tikz [baseline=-0.5ex]{\node [myarrow,rotate=90] {};}
}
\newcommand{\arrowdown}{%
\tikz [baseline=-1ex]{\node [myarrow,rotate=-90] {};}
}
\usepackage{hyperref}
\usepackage{natbib}
\usepackage{pgffor}
\usepackage{booktabs}
\usepackage{graphicx}
\usepackage{amssymb}
\usepackage{bbm}
\usepackage{tabularx}
\usepackage{tikz,etoolbox}
\usepackage{tikz,amsmath,siunitx}
\usetikzlibrary{arrows,snakes,backgrounds,patterns,matrix,shapes,fit,calc,shadows,plotmarks}

\usepackage{subcaption}
% \usepackage{url}
% \usepackage{hyperref}
\usepackage{pgf}
\usepackage{latexsym}
\usepackage{amsfonts}
\usepackage{amssymb}
\usepackage{amsthm}
\usepackage[noend]{algpseudocode}
\usepackage{algorithm}
\usepackage{amsmath}
\usepackage{tabularx}
\usepackage{xcolor}
\usepackage[absolute,overlay]{textpos}
\usetikzlibrary{shapes,arrows,positioning,automata,positioning,spy,matrix,scopes,chains}
\newcommand{\digs}[2]{\hphantom{999}\llap{#1}\,+\,\hphantom{999}\llap{#2}}
\setbeamersize{text margin left=6mm}
\setbeamersize{text margin right=6mm}
\renewcommand{\insertnavigation}[1]{}
\setbeamertemplate{headline}{}
\setbeamertemplate{footline}{}
\usefonttheme{professionalfonts}
\setbeamercovered{transparent}
\mode<presentation>
\linespread{1.25}
\DeclareMathOperator{\Tr}{Tr} 

\usepackage{color}
\usepackage{multirow}
\usepackage{rotating}
\usepackage[all,dvips]{xy}
\usepackage{colortbl}
\usepackage{graphicx}
\usepackage{verbatim}
\usepackage{framed}
\usepackage{natbib}
\usepackage[labelformat=empty]{caption}
\newcommand{\air}{\vspace{0.25cm}}
\newcommand{\mair}{\vspace{-0.25cm}}

\setbeamertemplate{navigation symbols}{}%remove navigation symbols
\renewcommand{\rmdefault}{crm}
\newcommand{\lnbrack}{{\normalfont [}}
\newcommand{\rnbrack}{{\normalfont ]}\thinspace}
\newcommand{\lbbrack}{\textcolor{red}{\textbf{[}}}
\newcommand{\rbbrack}{\textcolor{red}{\textbf{]}}\thinspace}
\definecolor{vermillion}{RGB}{213,94,0}
\newcommand{\given}{\,|\,}
\definecolor{orange}{RGB}{230,159,0}
\definecolor{skyblue}{RGB}{86,180,233}
\definecolor{bluegreen}{RGB}{90,143,41}
% \definecolor{bluegreen}{RGB}{0,158,115}
\definecolor{myyellow}{RGB}{240,228,66} % i dunno if this is the same as standard yellow
\definecolor{myblue}{RGB}{0,114,178}
\definecolor{vermillion}{RGB}{213,94,0}
\definecolor{redpurple}{RGB}{204,121,167}
\definecolor{lightgrey}{RGB}{234,234,234}

\newcommand{\ha}{\boldh_{\ua}}
\newcommand{\hp}{\boldh_{\up}}
\newcommand{\pgrad}{\nabla_{p}^\mathcal{L}}

\newcommand{\todoy}[1]{\textcolor{red}{Fix me (yoon): #1}}
\newcommand{\todos}[1]{\textcolor{red}{Fix me (sasha): #1}}

\newcommand{\hc}{\boldh_{\mathrm{c}}}
\DeclareMathOperator*{\argmax}{argmax}
\DeclareMathOperator*{\argmin}{argmin}
\DeclareMathOperator{\softmax}{softmax}
\DeclareMathOperator{\logadd}{logadd}
\DeclareMathOperator{\sign}{sign}
\DeclareMathOperator{\signexp}{signexp}
\DeclareMathOperator{\sigmoid}{sigmoid}
\DeclareMathOperator{\fwdbwd}{ForwardBackward}
\DeclareMathOperator{\softparent}{soft-parent}
\DeclareMathOperator{\parent}{parent}
\DeclareMathOperator{\head}{head}
\DeclareMathOperator{\softhead}{soft-head}
\DeclareMathOperator{\simf}{sim}
\DeclareMathOperator{\NN}{NNet}
\DeclareMathOperator{\attn}{Attn}
\DeclareMathOperator{\relu}{ReLU}
\DeclareMathOperator{\lstm}{LSTM}
\DeclareMathOperator{\rnn}{RNN}
\DeclareMathOperator{\mlp}{MLP}



\usetikzlibrary{positioning}
% \setbeamerfont{alerted text}{series=\bfseries}
% \setbeamerfont{structure}{series=\bfseries}
% Needed for diakgrams.
\def\im#1#2{
  \node(#1) [scale=#2]{\pgfbox[center,top]{\pgfuseimage{#1}}
};}
% \input{pictures_header}


\title[latent]{Title}

\author[us]{\today}
\institute[Harvard SEAS]{ 
{ }

 % Code: \textbf{https://github.com/harvardnlp/struct-attn}}
% \vspace{5mm}

% \hspace{-70mm} $^*$Equal Contribution
}
\date{}
% \usetheme{Madrid}
\definecolor{darkgreen}{rgb}{0.13, 0.55, 0.13}
\definecolor{darkpurple}{rgb}{0.55, 0.0, 0.55}

\newcommand{\enc}{\mathrm{src}}

\newcommand{\yvec}{\mathbf{y}}
\newcommand{\tvec}{\mathbf{t}}
\newcommand{\wvec}{\mathbf{w}}
\renewcommand{\c}{\mathbf{c}}
\newcommand{\zvec}{\mathbf{z}}
\newcommand{\E}{\mathbbm{E}}

\newcommand{\mcD}{\mathcal{D}}

% \newcommand{\mcY}{\mathcal{Y}}
% \newcommand{\mcV}{\mathcal{V}}
\newcommand{\context}{\mathbf{w}_{\mathrm{c}}}
\newcommand{\embcontext}{\mathbf{\tilde{w}}_{\mathrm{c}}}
\newcommand{\inpcontext}{\mathbf{\tilde{x}}}
\newcommand{\start}{\mathbf{\tilde{y}}_{\mathrm{c0}}}
\newcommand{\End}{\mathrm{\texttt{</s>}}}

\newcommand{\Uvec}{\mathbf{U}}
\newcommand{\Evec}{\mathbf{E}}
\newcommand{\Gvec}{\mathbf{G}}
\newcommand{\Fvec}{\mathbf{F}}
\newcommand{\Pvec}{\mathbf{P}}
\newcommand{\pvec}{\mathbf{p}}
\newcommand{\q}{\mathbf{Q}}
\newcommand{\Vvec}{\mathbf{V}}
\newcommand{\Wvec}{\mathbf{W}}
\newcommand{\h}{\mathbf{h}}
% \newcommand{\reals}{\mathbb{R}}

\newcommand{\Cite}[1]{{\footnotesize \citep{#1}}}
\newcommand{\TT}[1]{{\footnotesize\tt{#1}}}
\newcommand{\roplus}{{\color{red} \bigoplus}}
\newcommand{\rotimes}{{\color{red} \,\otimes\,}}

\newcommand{\boldw}{\boldsymbol{w}}
\newcommand{\xvec}{\mathbf{x}}

\newcommand{\boldu}{\boldsymbol{u}}
\newcommand{\boldv}{\boldsymbol{v}}
\newcommand{\boldb}{\boldsymbol{b}}
\newcommand{\boldW}{\boldsymbol{W}}
\newcommand{\boldh}{\boldsymbol{h}}
\newcommand{\boldg}{\boldsymbol{g}}
\newcommand{\ua}{\ensuremath{\mathrm{a}}}
\newcommand{\up}{\ensuremath{\mathrm{p}}}
%\newcommand{\bphi}{\ensuremath{\mathbf{\phi}}}
\newcommand{\bphi}{\boldsymbol{\phi}}
\newcommand{\btheta}{\boldsymbol{\theta}}
\newcommand{\mcY}{\mathcal{Y}}
\newcommand{\mcX}{\mathcal{X}}
\newcommand{\mcC}{\mathcal{C}}
\newcommand{\mcA}{\mathcal{A}}
\newcommand{\mcV}{\mathcal{V}}
\newcommand{\trans}{\ensuremath{\mathsf{T}}}
\def\argmin{\operatornamewithlimits{arg\,min}}
\def\argmax{\operatornamewithlimits{arg\,max}}
\newcommand{\reals}{\ensuremath{\mathbb{R}}}

\newcommand{\aphi}{\boldsymbol{\phi}_{\mathrm{a}}}
\newcommand{\pwphi}{\boldsymbol{\phi}_{\mathrm{p}}}
\newcommand{\squigaphi}{\widetilde{\boldsymbol{\phi}}_{\mathrm{a}}}
\newcommand{\squigpwphi}{\widetilde{\boldsymbol{\phi}}_{\mathrm{p}}}

\newcommand{\aW}{\boldW_{\mathrm{\ua}}}
\newcommand{\pW}{\boldW_{\mathrm{\up}}}

\newcommand{\ab}{\boldb_{\mathrm{\ua}}}
\newcommand{\pb}{\boldb_{\mathrm{\up}}}

\newcommand{\Da}{d_{\mathrm{a}}}
\newcommand{\Dp}{d_{\mathrm{p}}}

% \newcommand{\ha}{\boldh_{\ua}}
% \newcommand{\hp}{\boldh_{\up}}

\newcommand{\ourmodel}{This work}
\newcommand{\zro}{{\color{white}0}}
\AtBeginSection[]
{
  \begin{frame}
  \tableofcontents[currentsection]
  \end{frame}
}

\AtBeginSubsection[]
{
  \begin{frame}
  \tableofcontents[currentsubsection]
  \end{frame}
}


\def\argmax{\operatornamewithlimits{arg\,max}}
\def\kargmax{\operatornamewithlimits{K-arg\,max}}
\setbeamercovered{transparent}
\usepackage[absolute,overlay]{textpos}
\usepackage{animate}
\setbeamercovered{invisible}
\begin{document}
\begin{frame}
  \titlepage
\end{frame}

\begin{frame}
  \tableofcontents[hideallsubsections]
\end{frame}
\input{ssvae.tex}
\section{Section Title}
\begin{frame}{Frame Title}
    
\end{frame}
\section{Our Proposal: Attention as a Latent Variable}
\subsection{Attention Mechanism}
\begin{frame}{\subsecname: Basics}
    \begin{itemize}
        \item Source Input: $\mathbf{x} = \{x_1,x_2,\cdots, x_S\}$
        \item Target Output: $\mathbf{y} = \{y_1, y_2, \cdots, y_T\}$
        \item Encoded Source Features: $\text{enc}(\mathbf{x}) = \{\mathbf{h}_1, \mathbf{h}_2, \cdots, \mathbf{h}_S\}$
    \end{itemize}

    \begin{align*}
    \onslide<2->{P(\mathbf{y}|\mathbf{x}) & = \prod_{j=1}^T P(y_j | y_{<i}, \mathbf{x})\\}
    \onslide<3->{&=\prod_{j=1}^T P(y_j | y_{<j}, \text{enc}(\mathbf{x}))\\}
    \onslide<3->{&=\prod_{j=1}^T P(y_j | y_{<j}, \{\mathbf{h}_1, \mathbf{h}_2, \cdots, \mathbf{h}_S\})}
\end{align*}
    \end{frame}
\begin{frame}{\subsecname: Basics}
\begin{itemize}
    \item Recall that 
    \begin{align*}
    P(\mathbf{y}|\mathbf{x}) &=\prod_{j=1}^T P(y_j | y_{<j}, \{\mathbf{h}_1, \mathbf{h}_2, \cdots, \mathbf{h}_S\})
\end{align*}
    \item<2->$\{\mathbf{h}_1, \mathbf{h}_2, \cdots, \mathbf{h}_S\}$ is of varying lengths, in order to get fixed length vector, use weighted sum:
\begin{align*}
    P(y_j | y_{<j}, \{\mathbf{h}_1, \cdots, \mathbf{h}_S\})&=P(y_j | y_{<j}, \sum_{i=1}^S a_{ji} \mathbf{h}_i)
    \label{eq:main}
\end{align*}
    \item<3->The attention weights $\mathbf{a}_j = \{a_{j1}, a_{j2}, \cdots, a_{jS}\}$ lay on a simplex
    \item<4-> $\mathbf{a}_j$ are modeled (discriminatively) via a neural network $f$:
    \begin{equation*}
    \mathbf{a}_j = f(y_{<i}, \{\mathbf{h}_1, \mathbf{h}_2, \cdots, \mathbf{h}_S\})
\end{equation*}
\end{itemize}
\end{frame}

\begin{frame}{\subsecname: Interpretation as Alignments}
    \begin{itemize}
        \item Recall that
        \begin{align*}
        P(y_j | y_{<j}, \{\mathbf{h}_1, \cdots, \mathbf{h}_S\})&=P(y_j | y_{<j}, \sum_{i=1}^S a_{ji} \mathbf{h}_i)
        \label{eq:main}
        \end{align*}
        \item $P(y_j | y_{<j}, \mathbf{x})$ is related to $\mathbf{x}$ only through $\sum_{i=1}^S a_{ji} \mathbf{h}_i$
        \item The larger $a_{ji}$, the closer $\mathbf{h}_i$ to $\sum_{i=1}^S a_{ji} \mathbf{h}_i$
        \item $\mathbf{a}_j$ can be interpreted as how target word $j$ is aligned to source words $\{x_1,\cdots,x_S\}$
    \end{itemize}
\end{frame}
{
\usebackgroundtemplate{\vbox to \paperheight{\vfil\hbox to \paperwidth{\hfil\animategraphics[loop,controls,width=\paperwidth]{4}{img/recognition/recognition-}{5}{78}\hfil}\vfil}}
\begin{frame}[plain]
\end{frame}
}
\subsection{Attention as a Latent Variable}
\begin{frame}{\subsecname}
    \begin{itemize}
        \item<1-> Formally treat attention (alignments) as a latent variable
        \item<1-> Introduce prior over attention
        \item<2-> Still feed decoder with $\sum_{i=1}^S a_{ji} \mathbf{h}_i$
    \end{itemize}
        \begin{align*}
    \onslide<3->{&P(y_j, \mathbf{a}_j| y_{<j}, \mathbf{a}_{<j},\mathbf{x}) \\}
    \onslide<4->{&= P(y_j | y_{<j}, \mathbf{a}_{<j},\mathbf{x},\mathbf{a}_j)P(\mathbf{a}_j | y_{<j}, \mathbf{a}_{<j},\mathbf{x}  )\\}
    \onslide<5->{&= P(y_j | y_{<j}, \mathbf{a}_{<j},\sum_{i=1}^S a_{ji} \mathbf{h}_i)P(\mathbf{a}_j | y_{<j}, \mathbf{a}_{<j},\mathbf{x}  )\\}
    \end{align*}
\end{frame}

\begin{frame}{\subsecname}
    \begin{itemize}
        \item Recall that 
       \begin{align*}
    &P(y_j, \mathbf{a}_j| y_{<j}, \mathbf{a}_{<j},\mathbf{x}) \\
   &= P(y_j | y_{<j}, \mathbf{a}_{<j},\sum_{i=1}^S a_{ji} \mathbf{h}_i)P(\mathbf{a}_j | y_{<j}, \mathbf{a}_{<j},\mathbf{x}  )
    \end{align*}
    \item<2-> But we do not observe attentions $\mathbf{a}$
    \item<3-> We need to marginalize $\mathbf{a}$:
    \begin{align*}
    &\log P(\mathbf{y}|\mathbf{x}) = \log \int_{\mathbf{a}}\prod_{j=1}^T P(y_j | y_{<j}, \mathbf{a}_{<j},\sum_{i=1}^S a_{ji} \mathbf{h}_i)P(\mathbf{a}_j | y_{<j}, \mathbf{a}_{<j},\mathbf{x}  )
\end{align*}
    \item<4-> Normal attention is a special case if we set:
    \begin{align*}
        &P(\mathbf{a}_j | y_{<j}, \mathbf{a}_{<j},\mathbf{x})\\
        &=\delta(f(y_{<i}, \{\mathbf{h}_1, \mathbf{h}_2, \cdots, \mathbf{h}_S\}))
    \end{align*}
    \end{itemize}
   
\end{frame}

\begin{frame}{\subsecname: Training}
    \begin{itemize}
        \item Recall that 
    \begin{align*}
    &\log P(\mathbf{y}|\mathbf{x}) = \log \int_{\mathbf{a}}\prod_{j=1}^T P(y_j | y_{<j}, \mathbf{a}_{<j},\sum_{i=1}^S a_{ji} \mathbf{h}_i)P(\mathbf{a}_j | y_{<j}, \mathbf{a}_{<j},\mathbf{x}  )
\end{align*}
    \item<2-> Generally intractable due to the integral
    \item<3-> Use VAE by introducing an approximate posterior $Q(\mathbf{a}|\mathbf{y},\mathbf{x})$:
    \end{itemize}
    {\small
   \begin{align*}
 \onslide<4->{&\log P(\mathbf{y}|\mathbf{x}) \\}
  \onslide<5->{& = \log \int_{\mathbf{a}}Q(\mathbf{a})\frac{1}{{Q(\mathbf{a})}}\prod_{j=1}^T  P(y_j | y_{<j}, \mathbf{a}_{<j},\sum_{i=1}^S a_{ji} \mathbf{h}_i){P(\mathbf{a}_j | y_{<j}, \mathbf{a}_{<j},\mathbf{x})} \\}
  \onslide<6->{&\ge \int_{\mathbf{a}}Q(\mathbf{a})\log \frac{1}{{Q(\mathbf{a})}} \prod_{j=1}^T  P(y_j | y_{<j}, \mathbf{a}_{<j},\sum_{i=1}^S a_{ji} \mathbf{h}_i) {P(\mathbf{a}_j | y_{<j}, \mathbf{a}_{<j},\mathbf{x})} \\}
  \onslide<7->{&= \underset{\mathbf{a}\sim Q}{\mathbb{E}} [- \log Q(\mathbf{a})+\sum_{j=1}^T \log P(y_j | y_{<j}, \mathbf{a}_{<j},\sum_{i=1}^S a_{ji} \mathbf{h}_i)+ \log P(\mathbf{a}_j | y_{<j}, \mathbf{a}_{<j},\mathbf{x})]}
\end{align*}}
\end{frame}

\begin{frame}{\subsecname: Training}
    \begin{itemize}
        \item Recall that 
\end{itemize}
 {\small
   \begin{align*}
 &\log P(\mathbf{y}|\mathbf{x}) \\
 &= \underset{\mathbf{a}\sim Q}{\mathbb{E}} [- \log Q(\mathbf{a})+\sum_{j=1}^T \log P(y_j | y_{<j}, \mathbf{a}_{<j},\sum_{i=1}^S a_{ji} \mathbf{h}_i)+ \log P(\mathbf{a}_j | y_{<j}, \mathbf{a}_{<j},\mathbf{x})]\\
 &= \underbrace{\underset{\mathbf{a}\sim Q}{\mathbb{E}} [\sum_{j=1}^T \log P(y_j | y_{<j}, \mathbf{a}_{<j},\sum_{i=1}^S a_{ji} \mathbf{h}_i)]}_{\text{Fit data}}+ \underbrace{KL(Q(\mathbf{a})|| P(\mathbf{a}_j | y_{<j}, \mathbf{a}_{<j},\mathbf{x}))}_{Q(a) \text{ be close to prior }P(\mathbf{a}_j | y_{<j}, \mathbf{a}_{<j},\mathbf{x})}
\end{align*}}
    \begin{itemize}
        \item Apply reparamterization trick by specifying reparamterizable $Q(\mathbf{a}|\mathbf{y},\mathbf{x})$: Dirichlet or Logistic Normal
    \end{itemize}
\end{frame}

\begin{frame}{\subsecname: Training}
    \begin{itemize}
        \item<1-> Dirichlet:
        \begin{equation*}
            \mathbf{a}_j\sim \text{Dir}(\alpha_1,\cdots,\alpha_S)
        \end{equation*}
        \begin{itemize}
            \item Mean: $\frac{\alpha_1}{\alpha_0}, \cdots, \frac{\alpha_S}{\sum_i \alpha_0}$ where $\alpha_0=\sum_i \alpha_i$
            \item Variance: $\frac{\alpha_1(\alpha_0-\alpha_1)}{\alpha_0^2(\alpha_0+1)},\cdots,\frac{\alpha_S(\alpha_0-\alpha_S)}{\alpha_0^2(\alpha_0+1)}$
            \item Use inference network to generate the mean $\frac{\alpha_1}{\alpha_0}, \cdots, \frac{\alpha_S}{\sum_i \alpha_0}$ and $\alpha_0$.
        \end{itemize}
        \item<2-> Logistic Normal
        \begin{align*}
            &\mathbf{r}_j\sim \mathcal{N}(\mathbf{\mu}, \mathbf{\Sigma})\\
            &\mathbf{a}_j = \text{softmax}(\mathbf{r}_j)
        \end{align*}
        \begin{itemize}
            \item Use inference network to generate $\mathbf{\mu}$ and $\mathbf{\Sigma}$
        \end{itemize}
\end{itemize}
 
\end{frame}
\begin{frame}{\subsecname: Training}
\begin{itemize}
        \item Recall that 
\end{itemize}
 {\small
   \begin{align*}
 &\log P(\mathbf{y}|\mathbf{x}) \\
 &= \underset{\mathbf{a}\sim Q}{\mathbb{E}} [- \log Q(\mathbf{a})+\sum_{j=1}^T \log P(y_j | y_{<j}, \mathbf{a}_{<j},\sum_{i=1}^S a_{ji} \mathbf{h}_i)+ \log P(\mathbf{a}_j | y_{<j}, \mathbf{a}_{<j},\mathbf{x})]
\end{align*}}
    \begin{itemize}
        \item<2-> Sampling from $Q$ is equivalent to sampling from a simple distribution $\mathcal{U}$ followed by transformation $g_{\phi}$:
        \begin{align*}
            &\epsilon_j\sim \mathcal{U}\\
            &\mathbf{a}_j = g_\phi(\epsilon_j)
        \end{align*}
        \item<3-> The objective can be written as
        {\small
        \begin{equation*}
    \hspace{-1.3cm}\underset{\epsilon_1,\cdots,\epsilon_T\sim \mathcal{U}}{\mathbb{E}} \sum_{j=1}^T \log P(y_j | y_{<j}, \mathbf{a}_{<j},\sum_{i=1}^S (g_\phi (\epsilon_j))_{i} \mathbf{h}_i)+ \log P(g_\phi (\epsilon_j) | y_{<j}, \mathbf{a}_{<j},\mathbf{x}) + \mathcal{H}(Q)
\end{equation*}}
    \end{itemize}

\end{frame}

\begin{frame}{\subsecname: Inference}
\begin{itemize}
    \item Intractable to decode directly due to the integral w.r.t. $\mathbf{a}$
    \item First, we can find the joint argmax of $\mathbf{a}$ and $\mathbf{y}$ using beam search (stepwise) and back-propagation for optimizing w.r.t. $\mathbf{a}$
    \item Alternatively, we can use the model for rescoring
\end{itemize}
\end{frame}

\begin{frame}{\subsecname: Inference}
\begin{itemize}
    \item Intractable to decode directly due to the integral w.r.t. $\mathbf{a}$
    \item First, we can find the joint argmax of $\mathbf{a}$ and $\mathbf{y}$ using beam search (stepwise) and back-propagation for optimizing w.r.t. $\mathbf{a}$
    \item Alternatively, we can use the model for rescoring
\end{itemize}
\end{frame}

\begin{frame}{\subsecname: Network Structure}
\begin{itemize}
    \item $P(y_j | y_{<j}, \mathbf{a}_{<j},\sum_{i=1}^S a_{ji} \mathbf{h}_i)$: vanilla decoder
    \item $P(\mathbf{a}_j | y_{<j}, \mathbf{a}_{<j},\mathbf{x})$: vanilla attention network: the mean of $\mathbf{a}_j$, MLP with pooling over time: parameter controlling variance ($\sigma^2$ in logistic normal or $\alpha_0$ for Dirichlet).
    \item Inference network $Q$: normalized dot-products of embeddings: mean, MLP with pooling: variance
\end{itemize}
\end{frame}

\begin{frame}{\subsecname: Conclusion}
The proposed framework enjoys three-fold benefits:
\begin{itemize}
    \item alleviate decoder's burden to learn alignment model with a proper approximate attention posterior
    \item the attention network gets better supervision with the KL term 
    \item more flexible compared to vanilla attention
\end{itemize}
\end{frame}

\end{document}
