\documentclass{beamer}
% \mode<presentation>
\setbeamertemplate{navigation symbols}{}
\let\tempone\itemize
\let\temptwo\enditemize
\renewenvironment{itemize}{\tempone\addtolength{\itemsep}{0.5\baselineskip}}{\temptwo}
\usepackage{beamerthemeshadow}
\usepackage[normalem]{ulem}
\usepackage{tikz}
\usepackage{tikz-dependency}
\usetikzlibrary{shapes.arrows, positioning, fit, bayesnet}

\tikzset{
    myarrow/.style={
        draw,
        fill=gray,
        single arrow,
        minimum height=3.5ex,
        single arrow head extend=1ex
    }
}
\newcommand{\arrowup}{%
\tikz [baseline=-0.5ex]{\node [myarrow,rotate=90] {};}
}
\newcommand{\arrowdown}{%
\tikz [baseline=-1ex]{\node [myarrow,rotate=-90] {};}
}
\usepackage{hyperref}
\usepackage{natbib}
\usepackage{pgffor}
\usepackage{booktabs}
\usepackage{graphicx}
\usepackage{amssymb}
\usepackage{bbm}
\usepackage{tabularx}
\usepackage{tikz,etoolbox}
\usepackage{tikz,amsmath,siunitx}
\usetikzlibrary{arrows,snakes,backgrounds,patterns,matrix,shapes,fit,calc,shadows,plotmarks}
\usepackage{svg}

\usepackage{subcaption}
% \usepackage{url}
% \usepackage{hyperref}
\usepackage{pgf}
\usepackage{latexsym}
\usepackage{amsfonts}
\usepackage{amssymb}
\usepackage{amsthm}
\usepackage[noend]{algpseudocode}
\usepackage{algorithm}
\usepackage{amsmath}
\usepackage{tabularx}
\usepackage{xcolor}
\usepackage[absolute,overlay]{textpos}
\usetikzlibrary{shapes,arrows,positioning,automata,positioning,spy,matrix,scopes,chains}
\newcommand{\digs}[2]{\hphantom{999}\llap{#1}\,+\,\hphantom{999}\llap{#2}}
\setbeamersize{text margin left=6mm}
\setbeamersize{text margin right=6mm}
\renewcommand{\insertnavigation}[1]{}
\setbeamertemplate{headline}{}
\setbeamertemplate{footline}{}
\usefonttheme{professionalfonts}
\setbeamercovered{transparent}
\mode<presentation>
\linespread{1.25}
\DeclareMathOperator{\Tr}{Tr} 

\usepackage{color}
\usepackage{multirow}
\usepackage{rotating}
\usepackage[all,dvips]{xy}
\usepackage{colortbl}
\usepackage{graphicx}
\usepackage{verbatim}
\usepackage{framed}
\usepackage{natbib}
\usepackage[labelformat=empty]{caption}
\newcommand{\air}{\vspace{0.25cm}}
\newcommand{\mair}{\vspace{-0.25cm}}

\bibliographystyle{abbrvnat}

\setbeamertemplate{navigation symbols}{}%remove navigation symbols
\renewcommand{\rmdefault}{crm}
\newcommand{\lnbrack}{{\normalfont [}}
\newcommand{\rnbrack}{{\normalfont ]}\thinspace}
\newcommand{\lbbrack}{\textcolor{red}{\textbf{[}}}
\newcommand{\rbbrack}{\textcolor{red}{\textbf{]}}\thinspace}
\definecolor{vermillion}{RGB}{213,94,0}
\newcommand{\given}{\,|\,}
\definecolor{orange}{RGB}{230,159,0}
\definecolor{skyblue}{RGB}{86,180,233}
\definecolor{bluegreen}{RGB}{90,143,41}
% \definecolor{bluegreen}{RGB}{0,158,115}
\definecolor{myyellow}{RGB}{240,228,66} % i dunno if this is the same as standard yellow
\definecolor{myblue}{RGB}{0,114,178}
\definecolor{vermillion}{RGB}{213,94,0}
\definecolor{redpurple}{RGB}{204,121,167}
\definecolor{lightgrey}{RGB}{234,234,234}

\newcommand{\ha}{\boldh_{\ua}}
\newcommand{\hp}{\boldh_{\up}}
\newcommand{\pgrad}{\nabla_{p}^\mathcal{L}}

\newcommand{\todoy}[1]{\textcolor{red}{Fix me (yoon): #1}}
\newcommand{\todos}[1]{\textcolor{red}{Fix me (sasha): #1}}

\newcommand{\hc}{\boldh_{\mathrm{c}}}
\DeclareMathOperator*{\argmax}{argmax}
\DeclareMathOperator*{\argmin}{argmin}
\DeclareMathOperator{\softmax}{softmax}
\DeclareMathOperator{\logadd}{logadd}
\DeclareMathOperator{\sign}{sign}
\DeclareMathOperator{\signexp}{signexp}
\DeclareMathOperator{\sigmoid}{sigmoid}
\DeclareMathOperator{\fwdbwd}{ForwardBackward}
\DeclareMathOperator{\softparent}{soft-parent}
\DeclareMathOperator{\parent}{parent}
\DeclareMathOperator{\head}{head}
\DeclareMathOperator{\softhead}{soft-head}
\DeclareMathOperator{\simf}{sim}
\DeclareMathOperator{\NN}{NNet}
\DeclareMathOperator{\attn}{Attn}
\DeclareMathOperator{\relu}{ReLU}
\DeclareMathOperator{\lstm}{LSTM}
\DeclareMathOperator{\rnn}{RNN}
\DeclareMathOperator{\mlp}{MLP}



\usetikzlibrary{positioning}
% \setbeamerfont{alerted text}{series=\bfseries}
% \setbeamerfont{structure}{series=\bfseries}
% Needed for diakgrams.
\def\im#1#2{
  \node(#1) [scale=#2]{\pgfbox[center,top]{\pgfuseimage{#1}}
};}
% \input{pictures_header}


\title[latent]{Attention as a Latent Variable Model}

\author[Yoon Kim]{February 20, 2018}
\institute[Harvard SEAS]{ 
{ }

 % Code: \textbf{https://github.com/harvardnlp/struct-attn}}
% \vspace{5mm}

% \hspace{-70mm} $^*$Equal Contribution
}
\date{}
% \usetheme{Madrid}
\definecolor{darkgreen}{rgb}{0.13, 0.55, 0.13}
\definecolor{darkpurple}{rgb}{0.55, 0.0, 0.55}

\newcommand{\enc}{\mathrm{src}}

\newcommand{\yvec}{\mathbf{y}}
\newcommand{\tvec}{\mathbf{t}}
\newcommand{\wvec}{\mathbf{w}}
\renewcommand{\c}{\mathbf{c}}
\newcommand{\zvec}{\mathbf{z}}
\newcommand{\E}[2]{\mathbbm{E}_{#1}\left[#2\right]}

\newcommand{\mcD}{\mathcal{D}}

% \newcommand{\mcY}{\mathcal{Y}}
% \newcommand{\mcV}{\mathcal{V}}
\newcommand{\context}{\mathbf{w}_{\mathrm{c}}}
\newcommand{\embcontext}{\mathbf{\tilde{w}}_{\mathrm{c}}}
\newcommand{\inpcontext}{\mathbf{\tilde{x}}}
\newcommand{\start}{\mathbf{\tilde{y}}_{\mathrm{c0}}}
\newcommand{\End}{\mathrm{\texttt{</s>}}}

\newcommand{\Uvec}{\mathbf{U}}
\newcommand{\Evec}{\mathbf{E}}
\newcommand{\Gvec}{\mathbf{G}}
\newcommand{\Fvec}{\mathbf{F}}
\newcommand{\Pvec}{\mathbf{P}}
\newcommand{\pvec}{\mathbf{p}}
\newcommand{\q}{\mathbf{Q}}
\newcommand{\Vvec}{\mathbf{V}}
\newcommand{\Wvec}{\mathbf{W}}
\newcommand{\h}{\mathbf{h}}
% \newcommand{\reals}{\mathbb{R}}

\newcommand{\Cite}[1]{{\footnotesize \citep{#1}}}
\newcommand{\TT}[1]{{\footnotesize\tt{#1}}}
\newcommand{\roplus}{{\color{red} \bigoplus}}
\newcommand{\rotimes}{{\color{red} \,\otimes\,}}

\newcommand{\boldw}{\boldsymbol{w}}
\newcommand{\xvec}{\mathbf{x}}

\newcommand{\bx}{\mathbf{x}}
\newcommand{\by}{\mathbf{y}}
\newcommand{\bh}{\mathbf{h}}
\newcommand{\bc}{\mathbf{c}}
\newcommand{\bz}{\mathbf{z}}
\newcommand{\bs}{\mathbf{s}}
\newcommand{\bt}{\mathbf{t}}
\newcommand{\bd}{\mathbf{d}}
\newcommand{\bu}{\mathbf{u}}
\newcommand{\bv}{\mathbf{v}}

\newcommand{\boldu}{\boldsymbol{u}}
\newcommand{\boldv}{\boldsymbol{v}}
\newcommand{\boldb}{\boldsymbol{b}}
\newcommand{\boldW}{\boldsymbol{W}}
\newcommand{\boldh}{\boldsymbol{h}}
\newcommand{\boldg}{\boldsymbol{g}}
\newcommand{\ua}{\ensuremath{\mathrm{a}}}
\newcommand{\up}{\ensuremath{\mathrm{p}}}
%\newcommand{\bphi}{\ensuremath{\mathbf{\lambda}}}
\newcommand{\bphi}{\boldsymbol{\lambda}}
\newcommand{\btheta}{\boldsymbol{\theta}}
\newcommand{\mcY}{\mathcal{Y}}
\newcommand{\mcX}{\mathcal{X}}
\newcommand{\mcC}{\mathcal{C}}
\newcommand{\mcA}{\mathcal{A}}
\newcommand{\mcV}{\mathcal{V}}
\newcommand{\trans}{\ensuremath{\mathsf{T}}}
\def\argmin{\operatornamewithlimits{arg\,min}}
\def\argmax{\operatornamewithlimits{arg\,max}}
\newcommand{\reals}{\ensuremath{\mathbb{R}}}

\newcommand{\aphi}{\boldsymbol{\lambda}_{\mathrm{a}}}
\newcommand{\pwphi}{\boldsymbol{\lambda}_{\mathrm{p}}}
\newcommand{\squigaphi}{\widetilde{\boldsymbol{\lambda}}_{\mathrm{a}}}
\newcommand{\squigpwphi}{\widetilde{\boldsymbol{\lambda}}_{\mathrm{p}}}

\newcommand{\aW}{\boldW_{\mathrm{\ua}}}
\newcommand{\pW}{\boldW_{\mathrm{\up}}}

\newcommand{\ab}{\boldb_{\mathrm{\ua}}}
\newcommand{\pb}{\boldb_{\mathrm{\up}}}

\newcommand{\Da}{d_{\mathrm{a}}}
\newcommand{\Dp}{d_{\mathrm{p}}}

% \newcommand{\ha}{\boldh_{\ua}}
% \newcommand{\hp}{\boldh_{\up}}

\newcommand{\ourmodel}{This work}
\newcommand{\zro}{{\color{white}0}}
\AtBeginSection[]
{
  \begin{frame}
  \tableofcontents[currentsection]
  \end{frame}
}

\AtBeginSubsection[]
{
  \begin{frame}
  \tableofcontents[currentsubsection]
  \end{frame}
}


\def\argmax{\operatornamewithlimits{arg\,max}}
\def\kargmax{\operatornamewithlimits{K-arg\,max}}
\setbeamercovered{transparent}
\begin{document}

\begin{frame}
  \titlepage
\end{frame}

\begin{frame}
  \tableofcontents
\end{frame}
\section{Language as a Latent Variable}

\begin{frame}
  \begin{center}
    \structure{Overview of Paper}
   \end{center}   
   
\begin{itemize}
\item Semi-supervised summarization
\item Builds on \cite{kingma2014}'s M2 (the semisupervised version)
\item But uses a sequence of words as the latent representation
\end{itemize}
\end{frame}

\begin{frame}
\begin{center}
\structure{The Idea for Extractive Summarization}
\end{center}
\begin{itemize}
\item Start with source sentence $\bx =$ `I wish I could love dogs but I just hate them'
\item Sample a summary $\by =$ `I love dogs' by picking words from the source
\item Reconstruct the source sentence given the summary using an attentive decoder
\item Use the probability of the source sentence under the reconstruction decoder
(and a couple other terms) as signal for how good the summary was
\end{itemize}
\end{frame}

\begin{frame}
\begin{center}
\structure{Auto-Encoding Sentence Compression}
\end{center}
\center
\includegraphics[scale=0.5]{img/aec.pdf}
Note: Some of the parameters of the inference network
are used as the decoder network's encoder, but the shared parameters are not updated
using gradients from the decoder
\end{frame}

\begin{frame}
\begin{center}
\structure{Auto-Encoding Sentence Compression}
\end{center}
\begin{itemize}
\item The inference network $q_\lambda(\by\mid\bx)$
uses hard attention at every timestep to select a source token
\item Use a bidirectional source encoder on $\bx$ to get source embedding matrix
$H^e = \textrm{BRNN}(\bx)$, whose $i$th element is the vector $\bh_i^e$.
\item Select source token $\bx_i$ with a pointer network `compressor'
\begin{align}
\bh_j^c &= \textrm{RNN}(\bh_{j-1}^c, \by_{j-1})\\
\bu_j &= \textrm{attention}(\bh_j^c, H^e)\\
\by_j &\sim \textrm{Cat}(\bu_j)
\end{align}
\end{itemize}
\end{frame}

\begin{frame}
\begin{center}
\structure{ASC Continued}
\end{center}
\begin{itemize}
\item Let $H^c$ be the concatenation of all the compressor hidden states
\item The decoder $p_\theta(\bx\mid\by)$
is a conditional language model that attends over the hidden states $H^c$ 
to reconstruct the original source sentence $\bx$
\begin{align}
\bh_k^d &= \textrm{RNN}(\bh_{k-1}^d, \bx_{k-1})\\
\bv_k &= \textrm{attention}(\bh_k^d, H^c)\\
\bd_k &= \bv_k^TH^c\\
p_\theta(\bx_k\mid\bx_{<k}, \by) &= \textrm{softmax}(W \bd_k)
\end{align}
\item (7) is quite strange, since it does not use the recurrent state.
My guess is that it is a mistake from following the pointer network
paper too closely (this is equation (10) in the paper).
\end{itemize}
\end{frame}

\begin{frame}
\begin{center}
\structure{Details and Recap}
\end{center}
\begin{itemize}
\item The attention is the attention formulation from \cite{Vinyals2015a},
which is pretty close to the `general' attention in \cite{Luong2015}
\begin{equation}
\textrm{attention}(q, C) = \textrm{softmax}(\bv^T\textrm{tanh}(Wq + VC))
\end{equation}
\item $p_\theta(\bx\mid\by)$ is the reconstructive attention based decoder
\item $q_\lambda(\by\mid\bx)$ is the summarizing pointer network,
and we omit the conditioning on $\bx$ when convenient (randomly)
\item $p(\by)$ is a language model prior
\end{itemize}
\end{frame}

\begin{frame}
\begin{center}
\structure{Marginal Likelihood}
\end{center}
\begin{itemize}
\item The inference network's parameters will be denoted by $\lambda$
and the decoder network's by $\theta$
\item As usual, the marginal likelihood is intractable
\begin{equation}
\log p(\bx) = \log \sum_\by p_\theta(\bx, \by)
\end{equation}
since we cannot enumerate all possible summaries, even if they are extractive
\end{itemize}
\end{frame}

\begin{frame}
\begin{center}
\structure{Objective}
\end{center}
\begin{itemize}
\item So we lower bound it with Jensen's inequality and maximize the ELBO $\mathcal{L}$
\begin{align}
\log\sum_\by p(\bx, \by) &= \log\sum_\by q_\lambda(\by) \frac{p_\theta(\bx, \by)}{q_\lambda(\by)}\\
&= \log \E{\by\sim q_\lambda(\by)}{\frac{p_\theta(\bx, \by)}{q_\lambda(\by)}}\\
&\geq \E{\by\sim q_\lambda(\by)}{\log\frac{p_\theta(\bx, \by)}{q_\lambda(\by)}}\\
&= \E{\by\sim q_\lambda(\by)}{\log p_\theta(\bx\mid\by)} \\
&\qquad - D_{KL}(q_\lambda(\by) \Vert p(\by))\nonumber\\
&= \mathcal{L} \nonumber
\end{align}
\end{itemize}
\end{frame}

\begin{frame}
\begin{center}
\structure{Training Details}
\end{center}
\begin{itemize}
\item The gradient of the ELBO with respect to the reconstructive decoder only
depends on $p_\theta(\bx\mid\by)$
\begin{equation*}
\mathcal{L} = \underbrace{
        \E{\by\sim q_\lambda(\by)}{\log  p_\theta(\bx \mid \by)}
}_{\text{1. Reconstruction}} - \underbrace{
        KL[q_{\lambda^{}}(\zvec) \Vert p(\zvec^{})]
}_{\text{2. Regularization towards prior}}
\end{equation*}
\item It's given by term 1
\begin{align}
\nabla_\theta \mathcal{L} &=
\nabla_\theta \E{\by\sim q_\lambda(\by)}{\log  p_\theta(\bx \mid \by)}\\
&\approx \frac{1}{M}\sum_m \nabla_\theta\log p_\theta(\bx\mid\by^{(m)})
\end{align}
where $M$ sample summaries are generated through ancestral sampling
\end{itemize}
\end{frame}

\begin{frame}
\begin{center}
\structure{Training Details}
\end{center}
\begin{itemize}
\item The gradient with respect to the inference network requires REINFORCE
\item We rewrite the ELBO
\begin{align}
\mathcal{L} &= \E{\by\sim q_\lambda(\by)}{\log \frac{p_\theta(\bx,\by)}{q_\lambda(\by)}}\\
&= \E{\by\sim q_\lambda(\by)}{\log p_\theta(\bx\mid\by) + \log p(\by) - \log q_\lambda(\by)}\\
&= \E{\by\sim q_\lambda(\by)}{l(\bx,\by)}
\end{align}
\item So we have $l(\bx, \by) = \log p_\theta(\bx\mid\by) + \log p(\by) - \log q_\lambda(\by)$
\end{itemize}
\end{frame}

\begin{frame}
\begin{center}
\structure{REINFORCE}
\end{center}
\begin{itemize}
\item Recall the score function gradient estimator
\begin{equation}
p(\bx)\nabla \log p(\bx) = \nabla p(\bx)
\end{equation}
\item We use this to find an approximation of 
\begin{align}
\nabla_\lambda \mathcal{L} &= \nabla_\lambda\E{\by\sim q_\lambda(\by)}{l(\bx,\by)}\\
&= \sum_\by l(\bx,\by)\nabla_\lambda q_\lambda(\by)\\
&= \sum_\by q_\lambda(\by)l(\bx,\by)\nabla_\lambda\log q_\lambda(\by) )\\
&= \E{\by\sim q_\lambda(\by)}{l(\bx,\by) \nabla_\lambda\log q_\lambda(\by)}
\end{align}
\end{itemize}
\end{frame}

\begin{frame}
\begin{center}
\structure{Details}
\end{center}
\begin{itemize}
\item They also train a baseline to predict $l(\bx,\by)$ for variance reduction
\item They use a variant of KL annealing, and augment the loss as follows
$$l(\bx, \by) = \log p_\theta(\bx\mid\by) + \lambda(\log p(\by) - \log q_\lambda(\by))$$
with $\lambda = 0.1$ without justification, but note that increasing $\lambda$ results
in shorter summaries $\by$ since the prior $p(\by)$ prefers shorter sequences
\end{itemize}
\end{frame}


\begin{frame}
\begin{center}
\structure{Questions}
\end{center}
\begin{itemize}
\item Why don't we use
$$\log\sum_\by p(\by)p_\theta(\bx\mid\by) \geq \sum_\by \E{\by\sim p(\by)}{\log p_\theta(\bx\mid\by)}$$
instead of introducing $q_\lambda(\by)$ if we're using REINFORCE anyway?
\item Which parts of the reward
$$l(\bx, \by) = \log p_\theta(\bx\mid\by) + \log p(\by) - \log q_\lambda(\by\mid\bx)$$
can we decompose to try to lower variance a bit more?
\end{itemize}
\end{frame}


\begin{frame}
\begin{center}
\structure{Connection to MIXER}
\end{center}
\begin{itemize}
\item lol
\end{itemize}
\end{frame}


\begin{frame}
\begin{center}
\structure{Title}
\end{center}
\begin{itemize}
\item lol
\end{itemize}
\end{frame}

\bibliography{master}
\end{document}
